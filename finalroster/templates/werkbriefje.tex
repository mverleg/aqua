%% LaTeX report base template for two-column layout
% Mark Verleg, March 17th 2014

%% LaTeX TOP general header for style and packages
% Mark Verleg, January 19th 2014

% switch from some USA size to A4
\documentclass[a4paper,twoside]{article}

%% PACKAGES
% for missing packages, the lazy > 1GB way is: apt-get install texlive-full
% include packages
\usepackage{graphicx}	% for including images using \begin{figure}[ht] \includegraphics[keepaspectratio]{name.png}
\usepackage{amsmath}	% math formulas, with $ equation $ or \begin{equation}
\usepackage{multicol}	% to have selective areas, marged with \begin{multicols}{2}, have multiple columns
\usepackage{fancyhdr}	% allows adding headers and/or footers to the document
\usepackage{lastpage}	% get a reference to the last page (for page numbering)
\usepackage{adjustbox}	% scale things (pgf images) the same way as normal images
\usepackage{xcolor}	% provides \textcolor
\usepackage{layouts}	% just for outputting linewidths etc
\usepackage{ifthen}	% \setboolean for switching one/twoside
\usepackage{tikz}		% drawing library, related to pgf
\usepackage{pgf}		% to show .pgf images (as generated by matplotlib)
\usepackage{etoolbox}	% allows toggles and if statements
\usepackage[hidelinks]{hyperref}	% allows using simple URLs without infinite escape chars
\usepackage[backend=bibtex]{biblatex} % cite bibtex references
\usepackage{caption}	% align figure captions
\usepackage{braket}		% <a|b> notation, \Bra{}, \Braket{|}
\usepackage{url}		% get urls to display normally
% configure packages
\captionsetup[figure]{slc=off}	% align figure captions left
\newtoggle{electronic}
\newtoggle{finalversion}
\newtoggle{headers}

%% DEFINITIONS
% properties
\newcommand{\authorname}{Mark Verleg \textbackslash authorname}
\newcommand{\departmentname}{Department \textbackslash departmentname}
\newcommand{\departmentlogo}{}
\newcommand{\shorttitle}{Short title \textbackslash shorttitle}
\newcommand{\longtitle}{Long title \textbackslash longtitle}
%%\renewcommand{\abstract}{Abstract}
% functionality
\newcommand{\comment}[1]{}
\newcommand{\todo}[0]{\textcolor{red}{ TODO }}
\newcommand{\pgfimg}[1]{\includegraphics{#1.pdf}}
\newcommand{\topwidemarker}[0]{\par\noindent\rule{\dimexpr(0.5\textwidth-0.5\columnsep-0.4pt)}{0.4pt}\rule{0.4pt}{6pt}}
\newcommand{\bottomwidemarker}[0]{\vspace{\belowdisplayskip}\hfill\rule[-6pt]{0.4pt}{6.4pt}\rule{\dimexpr(0.5\textwidth-0.5\columnsep-1pt)}{0.4pt}}

%% ALTERNATIVES
% load images directly instead of through pdfs and makefile
%   \renewcommand{\pgfimg}[1]{\input{#1.pgf}}

%% DEFINITIONS
% properties
\renewcommand{\authorname}{Mark Verleg}
\renewcommand{\departmentname}{Library of Science}
\renewcommand{\departmentlogo}{}
\renewcommand{\shorttitle}{Werkbriefje}
\renewcommand{\longtitle}{Werkbriefje Campusdetachering}
% shortcuts

\usepackage[margin=0.5in]{geometry}
\usepackage{booktabs}
\pagenumbering{gobble}

\begin{document}
	
	%% CONTENT
	\section*{Werkbriefje}
		\paragraph{}
			
			Campus Detachering
			
			Gymnasion, Heijendaalseweg 141
			
			Postbus 9044, 6500 KD Nijmegen
			
			info@cd.ru.nl
			
			\url{www.ru.nl/studentenbanen}
			
		\paragraph{}
			
			Naam: {{ user.get_full_name }}
			
			Geboortedatum: {{ birthday }}
			
			Werkzaam: Library of Science
			
			Maand: {{ month_name|title }}
			
		\paragraph{}
			
			\begin{table}[!htb]
				\begin{tabular}{| l | l | r | r | r | l | l |} \hline
					\bfseries datum & \bfseries dag & \bfseries 100\% & \bfseries 140\% & \bfseries 175\% & \bfseries kostenplaats & \bfseries toelichting \\ \hline
					
					
						
							{{ day.day }} & {{ day.weekday }} & {{ day.hours }} & & & & \\ \hline
						
							{{ day.day }} & {{ day.weekday }} & & & & & \\ \hline
						
					
					
					\bfseries totaal & & \bfseries {{ total }} & \bfseries 0 & \bfseries 0 & & \\ \hline
				\end{tabular}
				\label{tab:working_hours}
			\end{table}
			
		\paragraph{}
			
			Afdeling: FNWI zaalwacht Kostenpl./pr. code/verb.: 
			
		\paragraph{}
			
			De declaratie is naar waarheid ingevuld.
			
			\begin{multicols}{2}
				\textbf{Werknemer}
				\\
				Naam: {{ user.get_full_name }}
				\\
				Handtekening:
				\\ 
				\\ 
				\\ 
				Datum: {{ date }}
				
			\columnbreak
				\textbf{Werkgever}
				\\
				Naam: Gonny Kremers
				\\
				Handtekening:
				\\ 
				\\ 
				\\
				Datum: 
				
			\end{multicols}
			
		\paragraph{}
			
			Campus Detachering betaalt elke maand uit. Werkbriefjes dienen uiterlijk de 4e van de volgende maand binnen te zijn om op tijd te kunnen worden verwerkt. {\bfseries Lever je werkbriefje dus op tijd in!}
			
\end{document}


